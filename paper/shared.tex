%\usepackage{natbib}
%\usepackage[sort]{cite}
\pdfoutput=1
%\usepackage[colorlinks=true,urlcolor=blue,citecolor=blue,linkcolor=blue]{hyperref}
\usepackage[english]{babel}
\usepackage[utf8]{inputenc}
\usepackage[T1]{fontenc}
\usepackage{amssymb}
\usepackage{tabularx}
\usepackage{quoting}
\usepackage{upquote}
\usepackage{subcaption}
\usepackage{multicol}
\usepackage{cancel}
\usepackage[framemethod=TikZ]{mdframed}
\usetikzlibrary{shapes}
\usetikzlibrary{snakes}
\usetikzlibrary{shapes.geometric}
\usepackage{wrapfig}
%\usepackage{caption}
%\usepackage[plain]{algorithmic}
\usepackage{algpseudocode}
\usepackage{rotating}
%\usepackage{cite}
\usepackage{booktabs}
%\usepackage{unicode-math}
%\usepackage{algorithm}% http://ctan.org/pkg/algorithm
%\usepackage{algpseudocode}% http://ctan.org/pkg/algpseudocode
\usepackage{xcolor}% http://ctan.org/pkg/xcolor
\makeatletter
\newsavebox{\@brx}
\newcommand{\llangle}[1][]{\savebox{\@brx}{\(\m@th{#1\langle}\)}%
  \mathopen{\copy\@brx\kern-0.5\wd\@brx\usebox{\@brx}}}
\newcommand{\rrangle}[1][]{\savebox{\@brx}{\(\m@th{#1\rangle}\)}%
  \mathclose{\copy\@brx\kern-0.5\wd\@brx\usebox{\@brx}}}
\makeatother
\usepackage{bbm}
\usepackage{jlcode}
\usepackage{graphicx}
\usepackage{amsmath,color}
\usepackage{mathrsfs}
\usepackage{float}
\usepackage[normalem]{ulem}
\usepackage{makecell}
\usepackage{indentfirst}
\usepackage{txfonts}
%\usepackage[epsilon, tsrm, altpo]{backnaur}

\makeatletter
\def\parsept#1#2#3{%
    \def\nospace##1{\zap@space##1 \@empty}%
    \def\rawparsept(##1,##2){%
        \edef#1{\nospace{##1}}%
        \edef#2{\nospace{##2}}%
    }%
    \expandafter\rawparsept#3%
}
\makeatother
\DeclareMathAlphabet{\mymathbb}{U}{BOONDOX-ds}{m}{n}
\newcommand{\listingcaption}[1]%
{%
\refstepcounter{lstlisting}\hfill%
Listing \thelstlisting: #1\hfill%\hfill%
}%
\newcolumntype{b}{X}
\newcolumntype{s}{>{\hsize=.7\hsize}X}
\usepackage{listings}
\lstset{
    language=Julia,
    basicstyle=\ttfamily\scriptsize,
    numberstyle=\scriptsize,
    % numbers=left,
    backgroundcolor=\color{gray!7},
    %backgroundcolor=\color{white},
    %frame=single,
    xleftmargin=2em,
    tabsize=2,
    rulecolor=\color{black!15},
    %title=\lstname,
    escapeinside={(*}{*)},
    breaklines=true,
    %breakatwhitespace=true,
    %framextopmargin=2pt,
    %framexbottommargin=2pt,
    frame=bt,
    extendedchars=true,
    inputencoding=utf8,
    columns=fullflexible,
    %escapeinside={(*@}{@*)},
}

\tolerance=1
\emergencystretch=\maxdimen
\hyphenpenalty=1000
\hbadness=1000

\makeatletter

%%%%%%%%%%%%%%%%%%%%%%%%%%%%%% User specified LaTeX commands.

%Journal reference.  Comma sets off: name, vol, page, year
\def\journal #1, #2, #3, 1#4#5#6{{\sl #1~}{\bf #2}, #3 (1#4#5#6) }
\def\pr{\journal Phys. Rev., }
\def\prb{\journal Phys. Rev. B, }
\def\prl{\journal Phys. Rev. Lett., }
\def\pl{\journal Phys. Lett., }
%\def\np{\journal Nucl. Phys., }


%%%%%%%%%%%%%%%%%%%%%%%%%%%%%%%%%%%%%%%%%%%%%%%%%%%%%%%%%%%%%%%%%%%%%%%%%%%%%%%%%%%%%%%%%%%%%%%%%%%%%%%%%%%%%%%%%%%%%%%%%%%%%%%%%%%%%%%%%%%%%%%%%%%%%%%%%%%%%%%%%%%%%%%%%%%%%%%%%%%%%%%%%%%%%%%%%%%%%%%%%%%%%%%%%%%%%%%%%%%%%%%%%%%%%%%%%%%%%%%%%%%%%%%%%%%%


%\usepackage[colorlinks, citecolor=blue]{hyperref}
\DeclareMathOperator*{\argmax}{arg\,max}

\newcommand{\eqname}[1]{\stepcounter{equation}\tag{\theequation : #1}}
%%%%%% Shortcut related
\newcommand{\<}{\langle}
\renewcommand{\>}{\rangle}
\newcommand{\out}{{\vx^L}}
\newcommand{\inp}{{\vx^0}}
\newcommand{\cquad}{{{ }_{\quad}}}
\newcommand{\pluseq}{\mathrel{+}=}
\newcommand{\minuseq}{\mathrel{-}=}
\newcommand{\vx}{{\mathbf{x}}}
\newcommand{\vg}{{\mathbf{g}}}
\newcommand{\vp}{{\mathbf{p}}}
\newcommand{\vy}{{\mathbf{y}}}
\newcommand{\Var}{{\mathrm{Var}}}
\newcommand{\Mean}{{\mathrm{E}}}
\newcommand{\vvalue}{{\texttt{value}}}
\newcommand{\grad}{{\texttt{grad}}}
\newcommand{\parameter}{{\texttt{parameter}}}
%%%%%% Convention related
\newcommand{\SWAP}{{\rm SWAP}}
\newcommand{\CNOT}{{\rm CNOT}}
\newcommand{\bigO}{{\mathcal{O}}}
\newcommand{\X}{{\rm X}}
\renewcommand{\H}{{\rm H}}
\newcommand{\Rx}{{\rm Rx}}
\renewcommand{\v}[1]{{\bf #1}}
\newcommand{\dataset}{{\mathcal{D}}}
\newcommand{\wfunc}{{\psi}}
\newcommand{\SU}{{\rm SU}}
\newcommand{\UU}{{\rm U}}
\newcommand{\thetav}{{\boldsymbol{\theta}}}
\newcommand{\gammav}{{\boldsymbol{\gamma}}}
\newcommand{\thetai}{{\theta^\alpha_l}}
\newcommand{\Expect}{{\mathbb{E}}}
\newcommand{\Tr}{{\rm Tr}}
\newcommand{\etc}{{\it etc~}}
\newcommand{\etal}{{\it etal~}}
\newcommand{\xset}{\mathbf{X}}
\newcommand{\fl}{\texttt{fl}}
\newcommand{\pdata}{\mathbf{\pi}}
\newcommand{\q}{\mathbf{q}}
\newcommand{\epdata}{\mathbf{\hat{\pi}}}
\newcommand{\gammaset}{\boldsymbol{\Gamma}}
\newcommand{\ei}{{\mathbf{e}_l^\alpha}}
\newcommand{\vtheta}{{\boldsymbol{\theta}}}
\newcommand{\sigmag}{{\nu}}
\newcommand{\sigmai}[2]{{\sigma^{#2}_{#1}}}
\newcommand{\qi}[1]{{q^{\alpha_{#1}}_{#1}}}
\newcommand{\BAS}{Bars-and-Stripes}
\newcommand{\circled}[1]{\raisebox{.5pt}{\textcircled{\raisebox{-.9pt} {#1}}}}
\newcommand{\qexpect}[1]{{\left\langle #1\right\rangle}}
\newcommand{\expect}[2]{{\mathop{\mathbb{E}}\limits_{\substack{#2}}\left[#1\right]}}
\newcommand{\var}[2]{{\mathop{\mathrm{Var}}\limits_{\substack{#2}}\left(#1\right)}}
\newcommand{\pshift}[1]{{p_{\thetav+#1}}}
\newcommand{\upcite}[1]{\textsuperscript{\cite{#1}}}
\newcommand{\bra}[1]{\mbox{$\left\langle #1 \right|$}}
\newcommand{\ket}[1]{\mbox{$\left| #1 \right\rangle$}}
\newcommand{\braket}[2]{\mbox{$\left\langle #1 | #2 \right\rangle$}}
\newcommand{\tr}[1]{\mathrm{tr}\mbox{$\left[ #1\right]$}}
\newcommand{\OPLUS}{{\oplus}}
\newcommand{\bigOPLUS}{{\bigoplus}}

\newcommand{\ra}[1]{\renewcommand{\arraystretch}{#1}}
\newcommand{\Inputs}{{\lambda}}
\newcommand{\Output}{{\lambda_o}}
\newcommand{\Tensors}{{\mathcal{T}}}
\newcommand{\cc}{cc({\mathcal{N}_{\rm IS}(G)})}

%%%%%% Comment related
\newcommand{\red}[1]{[{\bf  \color{red}{ST: #1}}]}
\newcommand{\xred}[1]{[{\bf  \color{red}{\sout{ST: #1}}}]}
\newcommand{\green}[1]{[{\bf  \color{green}{XG: #1}}]}
\newcommand{\xgreen}[1]{[{\bf  \color{green}{\sout{XG: #1}}}]}
\newcommand{\blue}[1]{[{\bf  \color{blue}{JG: #1}}]}
\newcommand{\xblue}[1]{[{\bf  \color{blue}{\sout{JG: #1}}}]}
\newcommand{\cyan}[1]{[{\bf  \color{cyan}{ML: #1}}]}
\newcommand{\xcyan}[1]{[{\bf  \color{cyan}{\sout{ML: #1}}}]}
\newcommand{\purple}[1]{[{\bf  \color{purple}{MC: #1}}]}
\newcommand{\xpurple}[1]{[{\bf  \color{purple}{\sout{MC: #1}}}]}
\newcommand{\material}[1]{\iffalse[{\bf  \color{cyan}{Material: #1}}]\fi}

\newcounter{example}
\newenvironment{example}[1][]{\refstepcounter{example}\par\medskip
   \noindent \textbf{Example~\theexample. #1} \rmfamily}{\medskip}

%\newtheorem{theorem}{\textit{Theorem}}
%\newtheorem{corollary}{\textit Branching Rule}
%\theoremstyle{definition}\newtheorem{definition}{\textit{Definition}}
%\newtheorem{defin}[thm]{Definition}

\makeatother

\author{
Jin-Guo Liu\thanks{Department of Physics, Harvard University, Cambridge, Massachusetts 02138, USA; QuEra Computing Inc., 1284 Soldiers Field Road, Boston, MA, 02135, USA  %\thanks{QuEra Computing Inc., 1284 Soldiers Field Road, Boston, MA, 02135, USA}
    (\email{jinguoliu@g.harvard.edu}).}
\and {Xun Gao}\thanks{Department of Physics, Harvard University, Cambridge, Massachusetts 02138, USA (\email{xungao@g.harvard.edu}, contributed equally with Jin-Guo Liu to this work.).}
\and Madelyn Cain\footnotemark[2]
\and Mikhail D. Lukin\footnotemark[2]
\and Sheng-Tao Wang\thanks{QuEra Computing Inc., 1284 Soldiers Field Road, Boston, MA, 02135, USA}
}
