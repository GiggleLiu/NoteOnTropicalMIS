%\documentclass[a4paper,superscriptaddress,11pt]{quantumarticle}
%\documentclass[aps,twocolumn,longbibliography,english,superscriptaddress]{revtex4-1}
\documentclass{article}
\usepackage{iclr2021_conference}
%\documentclass[a4paper,superscriptaddress,11pt]{article}
\pdfoutput=1
\usepackage[colorlinks=true,urlcolor=blue,citecolor=blue,linkcolor=blue]{hyperref}
\usepackage[english]{babel}
\usepackage[utf8]{inputenc}
\usepackage[T1]{fontenc}
\usepackage{amssymb}
\usepackage{tabularx}
\usepackage{quoting}
\usepackage{upquote}
\usepackage{subcaption}
\usepackage{multicol}
\usepackage[framemethod=TikZ]{mdframed}
\usepackage{wrapfig}
%\usepackage{caption}
%\usepackage[plain]{algorithm}
\usepackage[ruled, vlined]{algorithm2e}
\usepackage{algpseudocode}
\usepackage{rotating}
%\usepackage{cite}
\usepackage{booktabs}
%\usepackage{unicode-math}
%\usepackage{algorithm}% http://ctan.org/pkg/algorithm
%\usepackage{algpseudocode}% http://ctan.org/pkg/algpseudocode
\usepackage{xcolor}% http://ctan.org/pkg/xcolor
\makeatletter
\newsavebox{\@brx}
\newcommand{\llangle}[1][]{\savebox{\@brx}{\(\m@th{#1\langle}\)}%
  \mathopen{\copy\@brx\kern-0.5\wd\@brx\usebox{\@brx}}}
\newcommand{\rrangle}[1][]{\savebox{\@brx}{\(\m@th{#1\rangle}\)}%
  \mathclose{\copy\@brx\kern-0.5\wd\@brx\usebox{\@brx}}}
\makeatother
\usepackage{bbm}
\usepackage{jlcode}
\usepackage{graphicx}
\usepackage{amsmath,color,amsthm}
\usepackage{mathrsfs}
\usepackage{float}
\usepackage[normalem]{ulem}
\usepackage{makecell}
\usepackage{indentfirst}
\usepackage{txfonts}
\usepackage[epsilon, tsrm, altpo]{backnaur}

\newcommand{\listingcaption}[1]%
{%
\refstepcounter{lstlisting}\hfill%
Listing \thelstlisting: #1\hfill%\hfill%
}%
\newcolumntype{b}{X}
\newcolumntype{s}{>{\hsize=.7\hsize}X}
\usepackage{listings}
\lstset{
    language=Julia,
    basicstyle=\ttfamily\scriptsize,
    numberstyle=\scriptsize,
    % numbers=left,
    backgroundcolor=\color{gray!7},
    %backgroundcolor=\color{white},
    %frame=single,
    xleftmargin=2em,
    tabsize=2,
    rulecolor=\color{black!15},
    %title=\lstname,
    escapeinside={(*}{*)},
    breaklines=true,
    %breakatwhitespace=true,
    %framextopmargin=2pt,
    %framexbottommargin=2pt,
    frame=bt,
    extendedchars=true,
    inputencoding=utf8,
    columns=fullflexible,
    %escapeinside={(*@}{@*)},
}

\tolerance=1
\emergencystretch=\maxdimen
\hyphenpenalty=1000
\hbadness=1000

\makeatletter

%%%%%%%%%%%%%%%%%%%%%%%%%%%%%% User specified LaTeX commands.

%Journal reference.  Comma sets off: name, vol, page, year
\def\journal #1, #2, #3, 1#4#5#6{{\sl #1~}{\bf #2}, #3 (1#4#5#6) }
\def\pr{\journal Phys. Rev., }
\def\prb{\journal Phys. Rev. B, }
\def\prl{\journal Phys. Rev. Lett., }
\def\pl{\journal Phys. Lett., }
%\def\np{\journal Nucl. Phys., }


%%%%%%%%%%%%%%%%%%%%%%%%%%%%%%%%%%%%%%%%%%%%%%%%%%%%%%%%%%%%%%%%%%%%%%%%%%%%%%%%%%%%%%%%%%%%%%%%%%%%%%%%%%%%%%%%%%%%%%%%%%%%%%%%%%%%%%%%%%%%%%%%%%%%%%%%%%%%%%%%%%%%%%%%%%%%%%%%%%%%%%%%%%%%%%%%%%%%%%%%%%%%%%%%%%%%%%%%%%%%%%%%%%%%%%%%%%%%%%%%%%%%%%%%%%%%


%\usepackage{CJK}
%\usepackage[colorlinks, citecolor=blue]{hyperref}
\DeclareMathOperator*{\argmax}{arg\,max}

%%%%%% Shortcut related
\newcommand{\<}{\langle}
\renewcommand{\>}{\rangle}
\newcommand{\out}{{\vx^L}}
\newcommand{\inp}{{\vx^0}}
\newcommand{\cquad}{{{ }_{\quad}}}
\newcommand{\pluseq}{\mathrel{+}=}
\newcommand{\minuseq}{\mathrel{-}=}
\newcommand{\vx}{{\mathbf{x}}}
\newcommand{\vg}{{\mathbf{g}}}
\newcommand{\vp}{{\mathbf{p}}}
\newcommand{\vy}{{\mathbf{y}}}
\newcommand{\Var}{{\mathrm{Var}}}
\newcommand{\Mean}{{\mathrm{E}}}
\newcommand{\vvalue}{{\texttt{value}}}
\newcommand{\grad}{{\texttt{grad}}}
\newcommand{\parameter}{{\texttt{parameter}}}
%%%%%% Convention related
\newcommand{\SWAP}{{\rm SWAP}}
\newcommand{\CNOT}{{\rm CNOT}}
\newcommand{\bigO}{{\mathcal{O}}}
\newcommand{\X}{{\rm X}}
\renewcommand{\H}{{\rm H}}
\newcommand{\Rx}{{\rm Rx}}
\renewcommand{\v}[1]{{\bf #1}}
\newcommand{\dataset}{{\mathcal{D}}}
\newcommand{\wfunc}{{\psi}}
\newcommand{\SU}{{\rm SU}}
\newcommand{\UU}{{\rm U}}
\newcommand{\thetav}{{\boldsymbol{\theta}}}
\newcommand{\gammav}{{\boldsymbol{\gamma}}}
\newcommand{\thetai}{{\theta^\alpha_l}}
\newcommand{\Expect}{{\mathbb{E}}}
\newcommand{\Tr}{{\rm Tr}}
\renewcommand{\cite}[1]{{\citep{#1}}}
\newcommand{\etc}{{\it etc~}}
\newcommand{\etal}{{\it etal~}}
\newcommand{\xset}{\mathbf{X}}
\newcommand{\fl}{\texttt{fl}}
\newcommand{\pdata}{\mathbf{\pi}}
\newcommand{\q}{\mathbf{q}}
\newcommand{\epdata}{\mathbf{\hat{\pi}}}
\newcommand{\gammaset}{\boldsymbol{\Gamma}}
\newcommand{\ei}{{\mathbf{e}_l^\alpha}}
\newcommand{\vtheta}{{\boldsymbol{\theta}}}
\newcommand{\sigmag}{{\nu}}
\newcommand{\sigmai}[2]{{\sigma^{#2}_{#1}}}
\newcommand{\qi}[1]{{q^{\alpha_{#1}}_{#1}}}
\newcommand{\BAS}{Bars-and-Stripes}
\newcommand{\circled}[1]{\raisebox{.5pt}{\textcircled{\raisebox{-.9pt} {#1}}}}
\newcommand{\qexpect}[1]{{\left\langle #1\right\rangle}}
\newcommand{\expect}[2]{{\mathop{\mathbb{E}}\limits_{\substack{#2}}\left[#1\right]}}
\newcommand{\var}[2]{{\mathop{\mathrm{Var}}\limits_{\substack{#2}}\left(#1\right)}}
\newcommand{\pshift}[1]{{p_{\thetav+#1}}}
\newcommand{\upcite}[1]{\textsuperscript{\cite{#1}}}
\newcommand{\Eq}[1]{Eq.~(\ref{#1})}
\newcommand{\Fig}[1]{Fig.~\ref{#1}}
\newcommand{\Lst}[1]{Listing.~\ref{#1}}
\newcommand{\Tbl}[1]{Table~\ref{#1}}
\newcommand{\Sec}[1]{Sec.~\ref{#1}}
\newcommand{\App}[1]{Appendix~\ref{#1}}
\newcommand{\bra}[1]{\mbox{$\left\langle #1 \right|$}}
\newcommand{\ket}[1]{\mbox{$\left| #1 \right\rangle$}}
\newcommand{\braket}[2]{\mbox{$\left\langle #1 | #2 \right\rangle$}}
\newcommand{\tr}[1]{\mathrm{tr}\mbox{$\left[ #1\right]$}}

\newcommand{\ra}[1]{\renewcommand{\arraystretch}{#1}}

%%%%%% Comment related
\newcommand{\red}[1]{[{\bf  \color{red}{LW: #1}}]}
\newcommand{\xred}[1]{[{\bf  \color{red}{\sout{LW: #1}}}]}
\newcommand{\blue}[1]{[{\bf  \color{blue}{JG: #1}}]}
\newcommand{\violet}[1]{[{\bf  \color{violet}{MLS: #1}}]}
\newcommand{\green}[1]{[{\bf  \color{green}{TZ: #1}}]}
\newcommand{\xgreen}[1]{[{\bf  \color{green}{\sout{TZ: #1}}}]}
\newcommand{\xblue}[1]{[{\bf  \color{blue}{\sout{JG: #1}}}]}
\newcommand{\material}[1]{\iffalse[{\bf  \color{cyan}{Material: #1}}]\fi}
\newcommand{\orange}[1]{\iffalse[{\bf  \color{orange}{Jo: #1}}]\fi}

\newtheorem{theorem}{\textit{Rule}}
\theoremstyle{definition}\newtheorem{definition}{\textit{Definition}}

\makeatother

\iclrfinalcopy
\begin{document}
\title{Solving the maximum independant set problem with Tropical tensor networks}

\author{Jin-Guo Liu\\
Harvard University\\
\texttt{jinguoliu@harvard.edu}\\
%\AND
%Sheng-Tao Wang\\
%QuEra computing Inc.
}
\maketitle

\begin{abstract}
	Solving the maximum independent set size problem: the maximum independent set size, the degeneracy, the optimal configuration and the equivalence between different graphs.
\end{abstract}

\section{Tools}
\begin{description}
	\item[OMEinsum] a package for einsum \\ \href{https://github.com/under-Peter/OMEinsum.jl}{https://github.com/under-Peter/OMEinsum.jl}
	\item[TropicalGEMM] a package for efficient tropical matrix multiplication (compatible with OMEinsum) \\ \href{https://github.com/TensorBFS/TropicalGEMM.jl}{https://github.com/TensorBFS/TropicalGEMM.jl}
	\item[TropicalNumbers] a package providing tropical number types and tropical algebra, one o the dependency of TropicalGEMM \\ \href{https://github.com/TensorBFS/TropicalNumbers.jl}{https://github.com/TensorBFS/TropicalNumbers.jl}
\end{description}

\section{Computing degeneracy}
\section{Utilizing the sparsity}
Tensor network compression is an important tool to utilize sparsity.

We contract the tensors in a subregion $R \subseteq G$ of a graph $G$, and obtain a resulting tensor $A$ of rank $|C|$, where $C$ is the set of vertice tensors at the cut.
The maximum independant set size in this region with boundary configuration $\sigma \in \{0,1\}\otimes |C|$ is $A_{\sigma}$.
We say an entry $A_{\sigma_a}$ is ``better'' than $A_{\sigma_b}$ if
\begin{align}
(\sigma_a \land \sigma_b = \sigma_a) \land (A_{\sigma_a} \geq A_{\sigma_b}),\label{eq:compress}
\end{align}
where $\land$ is a bitwise and operations.
The first term means that whenever a bit in $\sigma_a$ has boolean value $1$, the corresponding bit in $\sigma_b$ is also $1$.
While the second term means the maximum independant set size with boundary configuration fixed to $\sigma_a$ is not less than that fixed to $\sigma_b$.
The word ``better'' means the best solution with boundary configuration $\sigma_a$ is never worse than that with $\sigma_b$.
When Eq. \ref{eq:compress} holds, It is easy to see that if $\sigma_b \cup \overline{\sigma_b}$ is one of the solutions for maximum independant sets in $G$, $\sigma_a \cup \overline{\sigma_b}$ is also a solution.


\subsection{The equivalence between branching and compression}
We are going the verify the Lemmas used in branching algorithm in book~\cite{Fomin2013}.
\begin{theorem} % basic
  If a vertex $v$ is in an independent set $I$, then none of its neighbors can be in $I$.
On the other hand, if $I$ is a maximum (and thus maximal) independent set,
and thus if $v$ is not in $I$ then at least one of its neighbors is in $I$.
\end{theorem}
\begin{theorem} % 2.6
Let $G=(V,E)$ be a graph, let $v$ and $w$ be adjacent vertices of $G$ such that $N[v] \subseteq N[w]$. Then
\begin{equation}
\alpha(G)=\alpha(G\backslash w).
\end{equation}
\end{theorem}

\begin{theorem} % 2.7
  Let $G = (V, E)$ be a graph and let $v$ be a vertex of $G$. If no maximum
independent set of $G$ contains $v$ then every maximum independent set of $G$ contains
at least two vertices of $N(v)$.
\end{theorem}

\begin{theorem} % 2.8
Let $G = (V, E)$ be a graph and $v$ a vertex of $G$. Then
\begin{equation}
\alpha(G) = \max(1 + \alpha(G \backslash N[v]), \alpha(G \backslash (M(v) \cup \{v\})).
\end{equation}
\end{theorem}

\begin{theorem} % 2.9
Let $G = (V, E)$ be a graph and $v$ be a vertex of $G$ such that $N[v]$ is a
clique. Then
\begin{equation}
\alpha(G) = 1 + \alpha(G \backslash N[v]).
\end{equation}
\end{theorem}


\begin{theorem}  %2.10
Let $G$ be a graph, let $S$ be a separator of $G$ and let $I(S)$ be the set of
all subsets of $S$ being an independent set of $G$. Then
\begin{equation}
\alpha(G) = \max_{A\in I(S)} |A| + \alpha(G \backslash (S \cup N[A])).
\end{equation}
\end{theorem}

\begin{theorem}  %2.11
Let $G = (V, E)$ be a disconnected graph and $C \subseteq V$ a connected com-
ponent of $G$. Then
\begin{equation}
\alpha(G) = \alpha(G[C]) + \alpha(G \backslash C)).
\end{equation}
\end{theorem}

\begin{align}
if |V| = 0 then
return 0
if \exists v \in V with d(v) \leq 1 then
return 1 + mis2(G \backslash N[v])
if \exists v \in V with d(v) = 2 then
(let u 1 and u 2 be the neighbors of v)
if {u 1 , u 2 } \in E then
return 1 + mis2(G \backslash N[v])
if {u 1 , u 2 } \in
/ E then
if |N 2 (v)| = 1 then
(let N 2 (v) = {w})
return max(2 + mis2(G \backslash (N 2 [v] \cup N[w])), 2 + mis2(G \backslash N 2 [v]))
if |N 2 (v)| > 1 then
return max(mis2(G \backslash N[v]), mis2(G \backslash (M(v) \cup cup))
if \exists v \in V with d(v) = 3 then
(let u 1 u 2 and u 3 be the neighbors of v)
if G[N(v)] has no edge then
if v has a mirror then
return max(1 + mis2(G \backslash N[v]), mis2(G \backslash (M(v) \cup {v}))
if v has no mirror then
return max(1 + mis2(G \backslash N[v]), 2 + mis2(G \backslash N[{u 1 , u 2 }]), 2 + mis2(G \backslash
(N[{u_1 , u_3 }] \cup \{u_2\})), 2 + mis2(G \backslash (N[{u_2 , u_3 }] \cup \{1\})))
if G[N(v)] has one or two edges then
return max(1 + mis2(G \backslash N[v]), mis2(G \backslash (M(v) \cup cup))
if G[N(v)] has three edges then
return 1 + mis2(G \backslash N[v])
if \Delta (G) \geq 6 then
choose a vertex v of maximum degree in G
return max(1 + mis2(G \backslash N[v]), mis2(G \backslash v))
if G is disconnected then
(let C \subseteq V be a component of G)
return mis2(G[C]) + mis2(G \backslash C)
if G is 4 or 5-regular then
choose any vertex v of G
return max(1 + mis2(G \backslash N[v]), mis2(G \backslash (M(v) \cup \{v\}))
if \Delta (G) = 5 and \delta (G) = 4 then
choose adjacent vertices v and w with d(v) = 5 and d(w) = 4 in G
return
\max(1 + mis2(G \backslash N[v]), 1 + mis2(G \backslash (\{v\} \cup M(v) \cup N[w])), mis2(G \backslash (M(v) \cup \{v, w\})))
\end{align}

\bibliographystyle{iclr2021_conference}
\bibliography{refs}

\appendix

\end{document}
